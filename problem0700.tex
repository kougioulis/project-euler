\documentclass[12pt]{article}
\usepackage[english]{babel}
\usepackage{amsmath}
\usepackage{amsfonts}
\usepackage{amsthm}

\newcommand{\Mod}[1]{\ (\mathrm{mod}\ #1)}

\newtheorem{proposition}{Proposition}[section]
\newtheorem{problem}{Problem}[section]

%\newcommand\Solution{%
%  \textbf{Solution:}\\%
%}
\begin{document}

\title{Project Euler Problem 700 - Eulercoin}
\date{}
\author{}
\maketitle

\textbf{Problem 700:} Leonhard Euler was born on 15 April 1707. Consider the sequence $1504170715041707 \cdot n \mod 4503599627370517$. An element of this sequence is defined to be an Eulercoin if it is strictly smaller than all previously found Eulercoins. \\

For example, the first term is $1504170715041707$ which is the first Eulercoin. The second term is $3008341430083414$ which is greater than $1504170715041707$ so is not an Eulercoin. However, the third term is $8912517754604$ which is small enough to be a new Eulercoin. \\

The sum of the first $2$ Eulercoins is therefore $1513083232796311$. Find the sum of all Eulercoins.


\begin{proof}

It is easy by iteratively setting values for $n \in \mathbb{Z}$ to find the first 14 Eulercoins (the 14th one being $e_{14} = 428410324$, after that being computationally unmanagable. \\

Let $a=1504170715041707, b=4503599627370517$ and $e$ denote the candidate Eulercoin. Notice that $(a,b) = 1$ so it's invertible in $\mathbb{Z}_b$, $~ \exists a^{-1}$ such that 

$$
a \cdot a^{-1} \Mod b \equiv 1 = e_1
$$

where $e_1$ is the second smallest Eulercoin (the first one being $e_0 = 0$). By the extended Euclidean Algorithm one finds $a^{-1} = 3451657199285664$.


\begin{equation}
\begin{split}
a\cdot n\Mod b = e \Rightarrow & an \equiv e\Mod b \Rightarrow \\
& n \equiv a^{-1} e\Mod b
\end{split}
\end{equation}

From this we can keep a list of the first 14 Eulercoins, start from the (second) smallest Eulercoin $e_1$ and iteratively increase the possible Eulercoin and if $n$ is smaller than all the previous Eulercoins, append to the list of Eulercoins, until we reach $e_{14}$.
 
\end{proof}


\end{document}